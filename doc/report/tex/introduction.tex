Il progetto \textbf{Evolution Simulator} consiste nella realizzazione di un simulatore di evoluzione naturale di entità chiamate \textit{blob}, in grado di muoversi e interagire con altre entità all'interno di un mondo.

Il comportamento di un \textit{blob} è definito dalla sua tipologia, che determina l'aggiornamento del suo stato all'iterazione successiva della simulazione e il risultato prodotto da una collisione con un'altra entità della simulazione, dipendente dalla tipologia di quest'ultima.

Per sopravvivere i blob devono muoversi all'interno del mondo della simulazione alla ricerca di \textit{cibo} nutriente per aumentare la loro vita corrente.

La proprietà di effetto del cibo con cui un blob collide determina le variazioni delle sue proprietà, oltre all'eventuale introduzione di ulteriori blob nel mondo della simulazione. 

La simulazione può inoltre presentare \textit{ostacoli} statici che, analogamente ai cibi, comporteranno delle modifiche allo stato del blob a seguito di una collisione.

I cibi vengono prodotti da entità chiamate \textit{piante}, che a intervalli regolari producono un cibo con un effetto di collisione dipendente dal comportamento della pianta.

La sessione di simulazione è inoltre caratterizzata da un valore di luminosità e un valore di temperatura variabili, che influenzano rispettivamente la dimensione del campo visivo e la velocità di tutti i blob presenti.

La quantità delle diverse entità, la luminosità, la temperatura e la quantità di giornate di cui è composta la simulazione sono parametrizzabili dall'utente attraverso un'apposita interfaccia.

Al termine della simulazione verranno rappresentati mediante grafici informazioni sulle entità osservate nelle diverse giornate in base alla loro categoria e alle loro proprietà.