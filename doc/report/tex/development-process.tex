\section{Processo di sviluppo}
È stata adottata una metodologia di sviluppo agile che introduce un metodo di Project Management che permette di realizzare un progetto per fasi, denominate \textit{sprint}, ognuna delle quali focalizzata su nuove funzioni. Al cliente, simulato dai membri del team, viene mostrato il lavoro svolto con continuità, per verificarne il gradimento, ed è quindi possibile apportare modifiche con estrema velocità. In questo modo si incrementa l’efficienza del gruppo di lavoro e dunque la produttività.

Si è fatto riferimento a \textbf{Scrum}, scegliendo tra i membri del team Daniele Giulianini come \textbf{Product Owner} e Alessandro Oliva come \textbf{Scrum Master} ed effettuando sprint settimanali per determinare i compiti da svolgere per ciascun membro del team e rivedere le funzionalità da implementare.

Il primo periodo del processo di sviluppo è stato caratterizzato dalla modellazione dell'architettura generale del progetto, attraverso la definizione dei requisiti e una prima progettazione mediante diagrammi delle classi UML, seguito dalla creazione del \textbf{Product Backlog} e da un primo \textbf{Sprint Planning}.

\subsection{Meeting}
I meeting tra i componenti del team sono avvenuti con frequenza regolare mediante la piattaforma Microsoft Teams. 
I membri del team hanno partecipato attivamente agli incontri svolgendo il ruolo di sviluppatori e dove necessario quelli di Scrum Master e Product Owner.
All'inizio di ogni settimana sono stati effettuati in modo congiunto la \textbf{Sprint Review} e lo \textbf{Spint Planning}. In questo tipo di meeting è stato passato in rassegna il lavoro svolto nella settimana appena conclusa e sono stati determinati gli obiettivi e i compiti da svolgere per ciascun membro del team. Ogni componente è stato aggiornato sullo stato di avanzamento dei task individuati e il Product Backlog è stato aggiornato di conseguenza. Altri incontri si sono svolti nel mezzo della settimana per aggiornare i membri del team sui progressi svolti e far emergere problematiche riscontrate allo scopo di studiare insieme delle soluzioni, similmente a come avverrebbe nel \textbf{Daily Scrum} a cui si è fatto riferimento.

\subsection{Divisione dei task}
All'avvio di ciascuno sprint settimanale è stato fatto utilizzo del product backlog per assegnare a ciascun componente del team una serie di task da svolgere durante la settimana, in modo da definire obiettivi che ogni membro del team è tenuto a portare a termine. Un task rappresenta uno o più requisiti tra quelli individuati nel primo periodo del processo di sviluppo.

\subsection{Revisione dei task}
Alla fine di ciascuno sprint, durante la \textbf{Sprint Review}, ciascun componente del team ha informato gli altri dei progressi nello svolgimento dei task a lui assegnati, notificando le varie difficoltà avute per lo svolgimento dei task. Sulla base del risultato della review è stato aggiornato e revisionato il Product Backlog, e sono stati valutati eventuali ritardi sulla tabella di marcia adottando le relative contromisure.

Generalmente al compimento di un task, prima della sua effettiva terminazione è stata fatta un'azione di code review dove il codice prodotto dal responsabile del task è stato mostrato a tutti i membri del team con l'obiettivo di individuare possibili refactor atti a migliorare la qualità del codice prodotto, diminuendo così potenziale debito tecnico.

\subsection{Scelta degli strumenti}
Sono stati utilizzati differenti tool a supporto del processo di sviluppo. L'obiettivo dell'utilizzo di tali tool è quello di servire gli sviluppatori durante tutto il processo di sviluppo automatizzandolo, con lo scopo di migliorarne l'efficienza e di consentire al gruppo di concentrarsi maggiormente sulla risoluzione dei requisiti del progetto stesso.

\begin{itemize}
    \item \textbf{SBT} come strumento di build automation, automatizzando una varietà di compiti software quotidiani durante lo sviluppo del progetto come la compilazione del codice sorgente.
    \item \textbf{Scalatest} per la scrittura ed esecuzione dei test automatizzati, adottando come convenzione l'utilizzo di \texttt{FunSpec};
    \item \textbf{Travis CI} come strumento di \textbf{Continuous Integration}, in modo da testare progetti software ospitati sul repository GitHub;
    \item \textbf{GitHub} come servizio di hosting del codice sorgente e file utilizzati durente il processo di sviluppo quali ad esempio il product backlog. 
\end{itemize}