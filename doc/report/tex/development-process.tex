\section{Processo di sviluppo}
È stata adottata una metodologia di sviluppo agile basata su \textbf{Scrum}, scegliendo tra i membri del team un \textbf{product owner} e uno \textbf{scrum master} e effettuando sprint settimanali per determinare i compiti da svolgere per ciascun membro del team.

Il primo periodo del processo di sviluppo è stato caratterizzato dalla modellazione dell'architettura generale del progetto, attraverso la definizione dei requisiti e lo sviluppo di grafici UML, seguito dalla creazione del \textbf{product backlog} e da un primo \textbf{sprint planning}.

\subsection{Divisione dei task}
All'avvio di ciascun sprint settimanale, è stato fatto utilizzo del product backlog per assegnare a ciascun componente del team una serie di task da svolgere durante la settimana.

\subsection{Meeting}
I meeting tra i componenti del team sono avvenuti con frequenza regolare mediante la piattaforma Microsoft Teams. La durata media di un incontro è stata di 2 ore.

\subsection{Revisione dei task}
Alla fine di ciascuno sprint è stata tenuta la \textbf{sprint review}, attraverso la quale ciascun componente del team ha informato gli altri dei progressi nello svolgimento dei task a lui assegnati.

Il product backlog è stato soggetto a più revisioni.

\subsection{Scelta degli strumenti}
Sono stati utilizzati i seguenti tool:

\begin{itemize}
    \item \textbf{SBT} come strumento di build automation;
    \item \textbf{Scalatest} per la scrittura ed esecuzione dei test automatizzati utilizzando diversi stili di testing;
    \item \textbf{Travis CI} come strumento per la \textbf{continuous integration}.
\end{itemize}