\section{Implementazione}
\subsection{Aspetti implementativi}

\subsection{Suddivisione del lavoro}
\subsubsection{Rei Beshiri}
Il mio ruolo nel progetto dal punto di vista implementativo riguarda principalmente lo sviluppo del model e delle sue diverse componenti in collaborazione con i membri del team, in particolare delle entità \textit{blob} del loro comportamento e reazione all'ambiente circostante e alle intersezioni tra le varie entità in gioco e degli effetti delle stesse così come della loro bounding box.

I test sviluppati riguardano le classi \code{BlobTest}, \code{DegradationTest}, \code{IntersectionTest}.

\subsubsection{Andrea Betti}
Ho contribuito insieme a Rei Beshiri all'implementazione dei comportamenti delle diverse entità della simulazione in \code{evo\_sim.model.EntityBehaviour} e agli effetti applicabili dalle entità \textit{Effectful} in \code{evo\_sim.model.effects.CollisionEffect}, realizzando le classi e le funzioni relative alle entità \textit{Food} e \textit{Plant} e contribuendo in misura minore ai comportamenti delle entità \textit{Blob} e all'estensione di \code{evo\_sim.model.EntityStructure}.

Ho inoltre implementato la logica di rappresentazione della simulazione utilizzando Java Swing all'interno della classe \code{evo\_sim.view.swing.custom.components.ShapesPanel}, utilizzata nella funzione \code{rendered} di \code{evo\_sim.view.swing.SwingView} realizzata da Alessandro Oliva.

Per quanto riguarda i test, ho realizzato \code{FoodTests} e \code{PlantTests}.
\subsubsection{Daniele Giulianini}

\subsubsection{Alessandro Oliva}

\subsubsection{Andrea Vaienti}