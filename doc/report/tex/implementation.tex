\section{Implementazione}
Il seguente capitolo motiva e dettaglia le scelte implementative ritenute rilevanti per una corretta comprensione del progetto.


\subsection{Utilizzo paradigma funzionale}
Sin dalle prime fasi di progettazione il team ha intrapreso la scelta di utilizzare il più possibile il paradigma funzionale, cercando di non ricorrere alle usuali abitudini di programmazione object-oriented. Per fare ciò sono state utilizzate diverse metodologie: 
\begin{itemize}
    \item inutilizzo dei side effect, creando ad ogni modifica un nuovo ogetto immutabile.
    \item utilizzo di funzioni ricorsive.
    \item utilizzo di funzioni higher-order che permettono una facile ed immediata realizzazione del parttern Strategy, consentono una maggiore riutilizzabilità del codice. In questo modo è possibile passare alle funzioni strategie esterne, non necessitando così di modificare il codice.
\end{itemize}

Il Controller e il Model dell'applicazione sono state realizzate con un approccio puramente funzionale, mentre la View adotta un approccio funzionale attraverso la libreria cats.effect.IO ove possibile.


Prof: "Ricordatevi che la lettura della relazione fino a: (i) i requirement, deve essere sufficiente per uno sviluppatore per giungere ad un sistema che fa quello che fa il vostro; (ii) al design, deve essere sufficiente per uno sviluppatore per giungere ad un sistema che in più è organizzato come il vostro; (iii) alla implementazione, è essenzialmente equivalente al vostro."
\subsubsection{View}
Bisogna parlare della View, in particolare dell'approccio funzionale utilizzato.

\subsubsection{Creazione di un nuovo world ad ogni iterazione}

\subsection{Utilizzo della programmazione logica}
Parlo del movimento

\subsection{Test}
Prof: "Cercate di dare una idea di quanto pensate che i vostri test automatizzati coprano il codice e dove: è importante per stimare il potenziale impatto di una modifica al software." 
Io faccio un discorso generale, se c'è un aspetto particolare nei vostri test scrivetelo pure.



\subsection{Suddivisione del lavoro}
Mia introduzione iniziale

\subsubsection{Persona qualsiasi}
Ognugno deve dire cosa ha fatto, la sezione di progetto riconducibile a lui (es: per Vaio l'IA e il disegno delle statistiche finali) e se ha collaborato con qualcuno.

\subsubsection{Rei Beshiri}
Il mio ruolo nel progetto dal punto di vista implementativo riguarda principalmente lo sviluppo del model e delle sue diverse componenti in collaborazione con i membri del team, in particolare delle entità \textit{blob} del loro comportamento e reazione all'ambiente circostante e alle intersezioni tra le varie entità in gioco e degli effetti delle stesse così come della loro bounding box.

I test sviluppati riguardano le classi \code{BlobTest}, \code{DegradationTest}, \code{IntersectionTest}.

\subsubsection{Andrea Betti}
Ho contribuito insieme a Rei Beshiri all'implementazione dei comportamenti delle diverse entità della simulazione in \code{evo\_sim.model.EntityBehaviour} e agli effetti applicabili dalle entità \textit{Effectful} in \code{evo\_sim.model.effects.CollisionEffect}, realizzando le classi e le funzioni relative alle entità \textit{Food} e \textit{Plant} e contribuendo in misura minore ai comportamenti delle entità \textit{Blob} e all'estensione di \code{evo\_sim.model.EntityStructure}.

Ho inoltre implementato la logica di rappresentazione della simulazione utilizzando \texttt{scala-swing} all'interno della classe \code{evo\_sim.view.swing.custom.components.ShapesPanel}, utilizzata nella funzione \code{rendered} di \code{evo\_sim.view.swing.SwingView} realizzata da Alessandro Oliva.

Per quanto riguarda i test, ho realizzato \code{FoodTests} e \code{PlantTests}.
\subsubsection{Daniele Giulianini}

\subsubsection{Alessandro Oliva}
Il mio contributo nel model è consistito nello sviluppo delle entità \textit{Obstacle}, le entità con status temporanei (il cui raffinamento è però stato curato da Rei Beshiri), in particolare lo \textit{SlowBlob}.

Mi sono occupato inoltre del ciclo giorno notte della simulazione, con rispettiva influenza sui blob in termini di velocità e campo visivo tramite appositi moduli di funzioni.

Per quanto riguarda la View, ne ho seguito lo sviluppo dalla prima versione in \texttt{ScalaFX} fino all'implementazione puramente funzionale attraverso il framework \texttt{Cats}, del quale con l'importante contributo di Daniele Giulianini è stato sviluppato un intero package che permette di utilizzare componenti Swing in maniera funzionale. Sempre con Daniele Giulianini mi sono occupato dell'integrazione fra View e Core mediante il framework monadico. Parallelamente a questa versione è stata sviluppata un interfaccia a linea di comando per permettere la fruizione dell'applicazione nella sua interezza anche mentre si lavorava alla View.

I test da me sviluppati sono inclusi in \code{ObstacleTests}.
\subsubsection{Andrea Vaienti}