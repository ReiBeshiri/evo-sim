\chapter{Conclusioni}
Il progetto è stata un'esperienza formativa facendoci imbattere con aspetti di gestione, progettazione e sviluppo di un sistema software complesso adottando un approccio professionale e curandone tutti gli aspetti dall'analisi al deployment al controllo della qualità durante tutto il processo di sviluppo.

Il lavoro è stato organizzato seguendo gli standard definiti dal framework Scrum e dall’utilizzo della metodologia \textit{agile}. Nel complesso, all’interno del team non sono state riscontrate grosse problematiche relative alla pianificazione degli Sprint e alla metodologia di sviluppo utilizzata seppur lontana dai canoni adottati fino a questo momento per lo sviluppo di un software.

Inoltre è stata una più che valida esplorazione del linguaggio Scala e delle sue potenzialità in termini di sviluppo e qualità software.

Interessante anche l'approccio dalla progettazione allo sviluppo del software con un nuovo paradigma, ovvero quello funzionale mai esplorato in altri progetti fatti finora.

In conclusione il team ha avuto un responso positivo, trovando un buon bilanciamento nella suddivisione dei lavori e una buona intesa tra i vari componenti del gruppo.
