\section{Retrospettiva}

\subsection{Avviamento}
La fase di avviamento si è incentrata sulla definizione dei requisiti di base del progetto, analisi del problema e definizione dei componenti principali del sistema. Sono stati stabiliti struttura e comportamento delle entità che popolano la simulazione e come avvengono le interazioni tra di queste. Queste operazioni hanno portato via molto tempo ma ci ha permesso di avere le idee chiare fin da subito, così da evitare di tornare sui propri passi per via di questioni definite in modo troppo opaco o generico. Buona parte della stesura della documentazione è stata prodotta in questa fase, così da consolidare termini, comportamento della simulazione e definire la struttura del sistema ad un grado sufficientemente fine di granulatirà. Sono stati anche preparati i tool di sviluppo, in particolare \textbf{SBT} e \textbf{Travis CI}. I diagrammi UML delle classi core del progetto sono stati prodotti in questa fase, in modo che si avesse una struttura solida di partenza a cui fare riferimento per lavorare in modo indipendente.

\subsection{Sprint 1}
Per avere un risultato percettibile del lavoro svolto si è deciso di partire da subito con lo sviluppo della View. Si è optato inizialmente per l'utilizzo di \texttt{ScalaFX} così da poter sfruttare la possibilità di utilizzare file di markup per il disegno dei componenti delle varie schermate e di conseguenza snellire il codice vero e proprio. Il prodotto finale dello sprint aveva già buona parte della struttura completa necessaria, nonostante nessuna funzione di rilievo fosse ancora presente.

\subsection{Sprint 2}
Nel corso dello sviluppo si è riconsiderato l'utilizzo di \texttt{ScalaFX} in quanto il design iniziale richiedeva che l'entry point dell'applicazione fosse nel controller. Inoltre si era prevista la view come modulo intercambiabile, e l'inizializzazione del toolkit del framework non permetteva un totale disaccoppiamento tra controller e view, nonostante tentativi di allacciamento dei due componenti che snaturavano l'utilizzo del framework per come è stato pensato. Alla luce di questi fatti si è optato per \texttt{scala-swing}, che al contrario si è rivelato più conforme alle specifiche imposte, al prezzo di dover scrivere codice anche per la parte grafica. Nel prodotto finale si disponeva della schermata di input e la schermata della simulazione con le entità, di cui i blob in movimento.

\subsection{Sprint 3}
A questo punto dello sviluppo l'enfasi si è spostata sulle entità e sul loro comportamento, includendo l'influenza del ciclo giorno/notte su di esse, il meccanismo di riproduzione dei blob e raffinamento della logica di movimento. Alla fine dello sprint le entità si muovevano secondo pattern precisi ed era possibile vedere il campo visivo delle entità attive. Sono stati pensati gli ostacoli, cibi e i diversi tipi di blob che sono stati implementati in parte alla fine di questo sprint.

\subsection{Sprint 4}
Sulla view si è voluto effetturare un ulteriore step in avanti, introducendo l'utilizzo del framework monadico \texttt{Cats} per avere un implementazione puramente funzionale di essa. Per enfatizzare la modularità della View si è sviluppata una seconda implementazione via linea di comando, inizialmente nata come alternativa rapida per testare il funzionamento della simulazione mentre si lavorava alla controparte in Swing. Durante la settimana è stata anche introdotta la logica di movimento in Prolog, che si è continuata a ottimizzare nel tempo. È stata ritoccata la logica di gestione delle collisioni e introdotta un entità pianta per lo spawn dei cibi nell'area della simulazione viste le iniziali difficoltà di gestione dello spawn di entità legate al risultato di collisioni.

\subsection{Sprint 5}
Il focus dell'ultimo sprint si è incentrato sulla rifattorizzazione del codice, e all'introduzione di ottimizzazioni per l'incremento delle performance. Fortunatamente gli imprevisti sono stati meno di quanto preventivato e la conclusione dei lavori è stata agevole. In questa fase è stato documentato il codice usando la \textbf{Scaladoc}.