\chapter{Requisiti}
Questo capitolo ha come scopo quello descrivere dettagliatamente tutti i requisiti del software implementato. La quasi totalità dei requisiti sono rimasti invariati sin dalle prime fasi del progetto, mentre alcuni sono stati leggermente modificati o eliminati. È bene precisare che qualunque requisito sottoelencato è stato selezionato in quanto verificabile.

\section{Requisiti di business}

L'applicazione dovrà disporre delle seguenti caratteristiche:
\begin{myEnumerate}
    \item[1] Business.
    \begin{myEnumerate}[label*=\arabic*.]
        \item[1.1] Visualizzazione di una simulazione di evoluzione naturale di entità \textit{blob} che hanno come obiettivo la sopravvivenza in un ambiente ostile;
        \item[1.2] I \textit{blob} si differenziano gli uni dagli altri in base a caratteristiche genetiche che ne influenzano il comportamento;
        \item[1.3] L'ambiente del simulatore sarà configurabile così da avere diversi scenari che influenzeranno l'andamento della simulazione.
    \end{myEnumerate}
\end{myEnumerate}

In particolare il simulatore funzionerà come segue:
\begin{itemize}
    \item Una simulazione è composta da diverse giornate in ognuna delle quali i \textit{blob} hanno l'obiettivo di procacciarsi del cibo che gli consenta di ottenere le energie necessarie a sopravvivere;
    \item I \textit{blob} per sopravvivere e riprodursi necessitano del cibo presente all'interno dell'ecosistema;
\end{itemize}

\section{Requisiti utente}
In particolare l'utente può usufruire dei seguenti aspetti:

\begin{myEnumerate}
    \item[2] Utente.
    \begin{myEnumerate}[label*=\arabic*.]
        \item[2.1] Parametrizzazione delle caratteristiche della simulazione tramite UI;
        \begin{myEnumerate}[label*=\arabic*.]
            \item[2.1.1] Cardinalità dei \textit{blob};
            \item[2.1.2] Cardinalità delle piante che producono il cibo;
            \item[2.1.3] Cardinalità massima delle giornate;
            \item[2.1.4] Cardinalità degli ostacoli presenti all'interno dell'ambiente;
            \item[2.1.5] Temperatura dell'ambiente;
            \item[2.1.6] Luminosità dell'ambiente;
        \end{myEnumerate}
        \item[2.2] Rappresentazione grafica dell'andamento della simulazione in 2D;
        \begin{myEnumerate}[label*=\arabic*.]
            \item[2.2.1] Rappresentazione delle entità all'interno dell'ambiente;
            \begin{myEnumerate}[label*=\arabic*.]
                \item[2.2.1.1] Rappresentazione dei \textit{blob};
		        \begin{myEnumerate}[label*=\arabic*.]
			        \item[2.2.1.1.2] Rappresentazione del campo visivo come circonferenza attorno al \textit{blob};
		        \end{myEnumerate}
                \item[2.2.1.2] Rappresentazione degli ostacoli;
                \item[2.2.1.3] Rappresentazione del cibo;
		\item[2.2.1.4] Rappresentazione delle piante;
            \end{myEnumerate}
            \item[2.2.2] Rappresentazione degli indicatori di evoluzione in real-time;
            \begin{myEnumerate}[label*=\arabic*.]
                \item[2.2.2.1] Rappresentazione della cardinalità della popolazione di \textit{blob} attuale;
		        \item[2.2.2.2] Rappresentazione della luminosità;
	    	    \item[2.2.2.3] Rappresentazione della temperatura;
                \item[2.2.2.4] Rappresentazione del giorno corrente;
            \end{myEnumerate}
	    \item[2.2.3] Rappresentazione dell'andamento della luminosità come indicatore cromatico sull'area della simulazione;
	    \item[2.2.4] Rappresentazione dell'andamento della temperatura come indicatore cromatico sull'area della simulazione;
        \end{myEnumerate}
	\item[2.3] Rappresentazione delle statistiche riguardanti l'andamento della simulazione;
	\begin{myEnumerate}[label*=\arabic*.]
		    \item[2.3.1] Rappresentazione della media dei valori delle caratteristiche genetiche dei \textit{blob} (velocità, percezione dell'ambiente, dimensione);
		    \item[2.3.2] Rappresentazione della cardinalità della popolazione di \textit{blob} e del cibo per ogni giornata;
		    \item[2.3.3] Rappresentazione della tipologia di entità che hanno composto la simulazione in percentuale rispetto al totale;
	    \end{myEnumerate}
	    \item[2.4] Rappresentazione della simulazione e delle statistiche finali visualizzata a schermo intero;
	    \item[2.5] Implementazione di una interfaccia testuale alternativa all'interfaccia grafica, equivalente in termini di espressività:
	    \begin{myEnumerate}
		\item[2.5.1] Lettura da console dei parametri definiti in 2.1;
		\item[2.5.2] Scrittura su console degli indicatori real-time definiti in 2.2.2;
		\item[2.5.3] Scrittura su console dei risultati della simulazione (velocità media dei \textit{blob}, dimensione media dei \textit{blob}, campo visivo medio dei \textit{blob}, numero di \textit{blob} sopravvissuti fino alla fine, quantità di cibo rimanente alla fine).
	    \end{myEnumerate}
	\end{myEnumerate}
\end{myEnumerate}

\section{Requisiti funzionali}
Il simulatore prodotto dovrà:

\begin{myEnumerate}
    \item[3] Funzionali.
    \begin{myEnumerate}[label*=\arabic*.]
	\item[3.1] Essere composto da giornate che si compongono di un numero fisso di iterazioni le quali sono l'unità temporale di aggiornamento della simulazione;
        \item[3.2] Supportare diverse tipologie di entità;
	\begin{myEnumerate}[label*=\arabic*.]
        	\item[3.2.1] \textit{blob};
		\begin{myEnumerate}[label*=\arabic*.]
        		\item[3.2.1.1] I \textit{blob} hanno delle caratteristiche genetiche di base;
			\begin{myEnumerate}[label*=\arabic*.]
        			\item[3.2.1.1.1] Velocità;
				\item[3.2.1.1.2] Campo visivo circolare;
				\item[3.2.1.1.3] Dimensione;
				\item[3.2.1.1.4] Vita;
    			\end{myEnumerate}
			\item[3.2.1.2] Possibilità per i \textit{blob} di disporre di abilità aggiuntive;
			\begin{myEnumerate}[label*=\arabic*.]
        			\item[3.2.1.2.1] Cannibalismo che implica la possibilità di nutrirsi di altri blob di dimensione inferiore alla propria per incrementare la vita;
				\item[3.2.1.2.2] Avvelenamento che implica una perdita maggiore di vita con l'avanzamento del tempo;
				\item[3.2.1.2.3] Rallentamento che implica una diminuzione della velocità di movimento;
    			\end{myEnumerate}
    		\end{myEnumerate}
		\item[3.2.2] Cibi il cui consumo provoca diversi effetti sui \textit{blob}
		\begin{myEnumerate}[label*=\arabic*.]
        		\item[3.2.2.1] Nutriente che implica l'aumento di vita attuale del \textit{blob};
			\item[3.2.2.2] Avvelenato che implica l'applicazione dello stato di rallentamento sul blob che vi collide;
			\item[3.2.2.3] Riproduttivo che implica la nascita di un nuovo blob, e un aumento della vita attuale del padre;
			\begin{myEnumerate}[label*=\arabic*.]
        			\item[3.2.2.3.1] Possibilità che un blob sia cannibale alla nascita;
    			\end{myEnumerate}
    		\end{myEnumerate}
		\item[3.2.3] Ostacoli la cui collisione impedisce il naturale movimento del \textit{blob};
		\begin{myEnumerate}[label*=\arabic*.]
        		\item[3.2.3.1] Dannoso che implica il decremento della vita del blob che vi collide;
			\item[3.2.3.2] Rallentatore di movimento che implica l'applicazione dello stato di rallentamento sul blob che vi collide;
    		\end{myEnumerate}
    	\end{myEnumerate}
	\item[3.3] Parametrizzazione delle diverse caratteristiche ambientali che influenzano tutti i \textit{blob};
	\begin{myEnumerate}[label*=\arabic*.]
        	\item[3.3.1] Temperatura che influenza la velocità di movimento;
		\item[3.3.2] Luminosità che influenza la percezione dell'ambiente;
    	\end{myEnumerate}
	\item[3.4] Movimento dei \textit{blob} all'interno dell'ambiente dettato da un AI;
	\begin{myEnumerate}[label*=\arabic*.]
        	\item[3.4.1] Movimento casuale finché non compare un entità commestibile all'interno del campo visivo;
		\item[3.4.2] Inseguimento dell'entità commestibile presente all'interno del campo visivo;
    	\end{myEnumerate}
	\item[3.5] Riproduzione dei \textit{blob} basata sulle caratteristiche genetiche del padre con una variazione randomica;
	\item[3.6] Possibilità di effettuare simulazioni con la possibilità per l'utente di specificare le caratteristiche della simulazione come descritto in 2.1;
	\item[3.7] Possibilità di visualizzare l'andamento della simulazione;	
	\begin{myEnumerate}[label*=\arabic*.]
		\item[3.7.1] Visualizzazione delle entità all'interno dell'ambiente come descritto in 2.2.1;
		\item[3.7.2] Visualizzazione degli indicatori di evoluzione in real-time come descritto in 2.2.2;
	\end{myEnumerate}
	\item[3.8] Possibilità di visualizzare i risultati finali della simulazione attraverso grafici;
	\end{myEnumerate}
\end{myEnumerate}

\section{Requisiti non funzionali}

\begin{myEnumerate}[label*=\arabic*.]
	\item[4] Non funzionali
	\begin{myEnumerate}[label*=\arabic*.]
		\label{sec:cpu}
		\item[4.1] Mantenimento di un framerate stabile e costante che si aggiri sui 60 FPS su una macchina con requisiti minimi di 4GB di RAM, CPU Dual Core da 2.8 GHz.
		\item[4.2] Interfaccia utente responsive;
	\end{myEnumerate}
\end{myEnumerate}


\section{Requisiti di implementazione}

\begin{myEnumerate}[label*=\arabic*.]
	\item[5] Implementazione
	\begin{myEnumerate}[label*=\arabic*.]
		\item[5.1] L'applicazione verrà sviluppata in Scala e Prolog, mentre per verificare la presenza di errori verranno implementati test in Scalatest.
	\end{myEnumerate}
\end{myEnumerate}